\documentclass{jlreq}
\usepackage{amsmath}
\usepackage{amssymb}
\usepackage{amsfonts}
\usepackage{amsthm} 
\usepackage{macros} 

\usepackage[backend=biber,style=alphabetic,url=false]{biblatex}
\addbibresource{main.bib}

\title{テンプレート}
\author{Mateusz Zunderewski}
\date{\today}

\begin{document}

\maketitle

\begin{abstract}
	テンプレート
\end{abstract}

\setcounter{tocdepth}{3}
\tableofcontents

\pagebreak

Modal Companionのより発展的な話題については\cite*{chagrov_modal_1992}を参照されたい.

\begin{definition}[Gödel変換]\index{げーでるへんかん@Gödel変換}
	命題論理式を様相論理式へ写すGödel変換 $\cdot^G$ という操作を以下のように定義する.
\end{definition}

直観主義命題論理 $\LogicInt$ と様相論理 $\LogicSFour$ について次の定理\ref{thm:GMT}が知られている.

\begin{theorem}[Gödel-McKinsey-Tarskiの定理\cite*{godel_interpretation_1933,mckinsey_theorems_1948}]\label{thm:GMT}
	\begin{equation*}
		\LogicInt \vdash A \iff \LogicSFour \vdash A^G
	\end{equation*}
\end{theorem}

定理\ref{thm:GMT}をより一般化した関係を考える.

\begin{definition}[中間論理]\index{ちゅうかんろんり@中間論理}
	直観主義命題論理 $\LogicInt$ と古典命題論理 $\LogicCl$ の間の強さの論理を中間論理と呼ぶ.
	ただし $\LogicInt$ と $\LogicCl$ も中間論理に含める.
	すなわち,命題論理 $\Logic{I\Lambda}$ が中間論理とは $\LogicInt \leq \Logic{I\Lambda} \leq \LogicCl$ が成り立つことである.
\end{definition}

\begin{definition}[Modal Companion]\index{Modal Companion}
	様相論理 $\Logic{M\Lambda}$ が中間論理 $\Logic{I\Lambda}$ のModal Companionであるとは,
	任意の様相論理式 $A$ に対して以下が成り立つこととする.
	\begin{equation*}
		\Logic{I\Lambda} \vdash A \iff \Logic{M\Lambda} \vdash A^G
	\end{equation*}
\end{definition}

この定義よりGödel-McKensey-Tarskiの定理は「$\LogicSFour$ は $\LogicInt$ のModal Companionである」とも言い換えられる.


系\ref{cor:S4_weakerThan_Grz}より $\LogicSFourGrz \vdash A \iff \LogicGrz \vdash A \implies \LogicSFour \vdash A$ であるから,
$\LogicSFourGrz \vdash A^G \implies \LogicInt \vdash A$ が成り立つ.
実はこの逆も成り立つことがわかっているため,次のことが成り立つ.

\begin{theorem}[\cite*{grzegorczyk_relational_1969}]
	$\LogicSFourGrz$ も $\LogicInt$ のModal Companionである.
\end{theorem}

よって $\LogicInt$ のModal Companionとなる様相論理は複数存在する.
そしてこれは $\LogicInt$ だけに限らず一般の中間論理に対しても同様に,なおかつ次の強い主張が成り立つことも知られている.

\begin{theorem}[\cite*{maksimova_lattice_1974}]
	任意の中間論理 $\Logic{I\Lambda}$ のModal Companionは無限個存在し,かつ $\leq$ によって束を形成する.
\end{theorem}

そのような束の下限と上限,つまり各論理のModal Companionの中で最弱と最強の論理を考えたい.

\begin{definition}
	中間論理 $\Logic{I\Lambda}$ とし,その任意のModal Companionを $\Logic{M\Lambda}$ とする.

	\begin{enumerate}
		\item
		      $\tau\Logic{I\Lambda} \leq \Logic{M\Lambda}$ が成り立つような
		      最弱の $\Logic{I\Lambda}$ のModal Companionを $\tau\Logic{I\Lambda}$ とする.
		\item
		      $\Logic{M\Lambda} \leq \sigma\Logic{I\Lambda}$ が成り立つような
		      最強の $\Logic{I\Lambda}$ のModal Companionを $\sigma\Logic{I\Lambda}$ とする.
	\end{enumerate}
\end{definition}

\begin{theorem}[\cite*{dummett_modal_1959}]
	中間論理 $\Logic{I\Lambda}$ に対して,$\tau \Logic{I\Lambda} = \LogicSFour \oplus \{ A^G \mid \Logic{I\Lambda} \vdash A \} $ である.
\end{theorem}

\begin{corollary}
	$\tau \LogicInt = \LogicSFour$ である.
\end{corollary}

\begin{theorem}[Blok-Esakiaの定理\cite*{blok_varieties_1976,esakia_theory_1979,esakia_varieties_1979}]
	中間論理 $\Logic{I\Lambda}$ に対して,$\sigma \Logic{I\Lambda} = \tau\Logic{I\Lambda} \oplus \AxGrz$ である.
\end{theorem}

\begin{corollary}
	$\sigma \LogicInt = \LogicSFourGrz$ である.
\end{corollary}

\printbibliography

% \printindex

\end{document}